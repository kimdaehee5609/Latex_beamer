% -*- TeX:UTF-8 -*-
% &pdflatex, projectmain = Beamer_sample.tex
%
% (c) 2008, Eung-Shin Lee
% public domain.
%
\documentclass[10pt,compress,slidetop,%
			   hyperref={unicode},xcolor={svgnames},% sub-package options
			   t]{beamer}
% 이곳에서는 글씨체의 크기를 10 pt로 하였음, t는 본문 내용이 슬라이드의 위(top)부터 시작
%
% Beamer 저작권
% Copyright 2004      by Till Tantau  <tantau@users.sourceforge.net>.
%       and 2005-2006 by Jorg Cassens <jorg.cassens@idi.ntnu.no>
%


\mode<presentation> 									% 출력 양식을 프레젠테이션으로
{
	\usetheme{Frankfurt}      							% 템플릿은 프랑크푸르트로

% 템플릿 종류: Antibes, Bergen, Berkerley, Berlin, Boadilla, Copenhagen, Darmstadt, Dresden, Frankfurt, Goettingen,
%  Hannover, Ilmenau, JuanLesPins, Luebeck, Madrid, Malmoe, Marburg, Montpellier, PaloAlto, Pittsburgh, Rochester,
%  Singapore, Szeged, Warsaw
%
	\usecolortheme[named=olive]{structure}		% 색상 결정
 	\usefonttheme[onlymath]{serif}					% 수학공식에서 사용하는 글씨체
  % \useoutertheme{infolines}
  % or whatever

  \setbeamercovered{transparent}						% overlay 사용
  % or whatever (possibly just delete it)
}

%==================  필요한 패키지 설정 ===============
\usepackage{verbatim}
\usepackage{pdfpages}
\usepackage{multimedia}
\usepackage{animate}
%
\usepackage{bm}% bold math
\usepackage{amsmath,amssymb,amscd}
\usepackage{subeqnarray}
\usepackage{mathptmx}
%
\usepackage{helvet}
\usepackage{courier}
\usepackage{relsize}
%
\usepackage{colortbl}
\usepackage{wasysym}
%
\usepackage{tikz}
\usetikzlibrary{arrows,shapes,calc,patterns}
\tikzstyle{every picture}+=[remember picture]
\usetikzlibrary{%
    decorations.pathreplacing,%
    decorations.pathmorphing%
}
% ================== 한글 사용 =====================
\usepackage{dhucs}
\usepackage{ifpdf}
\ifpdf
  \input glyphtounicode\pdfgentounicode=1
\fi
%==============================================
%
%================ dual screen =======================
%\setbeameroption{show notes on second screen=left}
%\setbeamertemplate{note page}[compress]
%==============================================
%
%================ 제목에 사용할 글씨체 설정 ==============
\setbeamerfont{title}{shape=\itshape,family=\rmfamily}%
\setbeamerfont{frametitle}{family=\rmfamily}
%\usepackage[T1]{fontenc}
% Or whatever. Note that the encoding and the font should match. If T1
% does not look nice, try deleting the line with the fontenc.
%===============================================
%
%==================  표지 만들기 =====================
\title % (optional, use only with long paper titles)
{Beamer 결론 예제}
\subtitle{}


\author{홍길동 }													% 저자(발표자)

% Till Tantau\author{{1} \and
% J\"{o}rg Cassens\inst{2}

\institute[] 															% (optional, but mostly needed) 소속기관 약자 표기
{자신의 소속 기관}													% 표지에 나타나는 소속기관 full name
% - Use the \inst command only if there are several affiliations.
% - Keep it simple, no one is interested in your street address.

\date[25-03-11] 														% (optional, should be abbreviation of conference name)
{}																			% 날짜 지정: \today
% - Either use conference name or its abbreviation.
% - Not really informative to the audience, more for people (including
%   yourself) who are reading the slides online

\subject{Beamer}
% This is only inserted into the PDF information catalog. Can be left out.

% Delete this, if you do not want the table of contents to pop up at
% the beginning of each subsection:
%===============================================
%
% ====================  차 례 ======================
 \AtBeginSection[]
 {
 \begin{frame}<beamer>
    \frametitle{차 례}
    \tableofcontents[currentsection,hideallsubsections]
    %\tableofcontents[currentsection,currentsubsections]
  \end{frame}
}
%===============================================
%
%
%==========================  본문 시작 ==============================
\begin{document}
%
%=================  표지를  화면으로 나타나기 =============
\begin{frame}
  \titlepage
\end{frame}
%===============================================
%
% ******************************
\section{Introduction}
%*******************************
%
%------------------------------- 슬라이드  --------------------------------------------------
\begin{frame}
	\frametitle{}
	
\end{frame}
%----------------------------------------------------------------------------------------------
%
%-------------------------------- 슬라이드  -------------------------------------------------
\begin{frame}
	\frametitle{}
	
\end{frame}
%----------------------------------------------------------------------------------------------
%
%*******************************
\section{본문 내용}
%*******************************
%
%------------------------------- 슬라이드  --------------------------------------------------
\begin{frame}
	\frametitle{}
	
\end{frame}
%----------------------------------------------------------------------------------------------
%
%------------------------------- 슬라이드  --------------------------------------------------
\begin{frame}
	\frametitle{}
	
\end{frame}
%----------------------------------------------------------------------------------------------
%
%------------------------------- 슬라이드  --------------------------------------------------
\begin{frame}
	\frametitle{}
	
\end{frame}
%----------------------------------------------------------------------------------------------
%
%******************************
\section{결론}
%******************************
%
%------------------------------- 슬라이드  --------------------------------------------------
\frame<1>[label=conclusion]
{	\frametitle{결론}
\begin{enumerate}
	\item<alert@1> 첫 번째 결론
	\item<alert@2> 두 번째 결론
	\item<alert@3> 세 번째 결론
\end{enumerate}
}
\frame	
{
\includegraphics[width=0.8\textwidth]{dino1.jpg}
}
\againframe<2>{conclusion}
\frame
{
\includegraphics[width=0.8\textwidth]{dino2.jpg}
}
\againframe<3>{conclusion}
\frame
{
\includegraphics[width=0.4\textwidth]{dino3.jpg}
}
%----------------------------------------------------------------------------------------------
%
%
\end{document}
%==========================  본문 끝 = ================================